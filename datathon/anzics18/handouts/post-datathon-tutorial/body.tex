\begin{center}

\begin{huge}
\textsc{Working After the Datathon with Google Cloud}
\end{huge}

\vspace{3mm}

\begin{Large}
ANZICS Core Critical Care Datathon
\end{Large}
\end{center}

\section*{Creating Google Cloud Project}
Google Cloud Platform Console: https://console.cloud.google.com/
\begin{itemize}
\item Sign up for a free one-year trial, with \$300 credit.
\item We offer datathon participants an additional \$200 credit coupon in the
  box below.
  \begin{center}
    \fbox{\hspace{1cm} COUPON-CODE \hspace{1cm}}
  \end{center}
  This coupon can be redeemed at https://console.cloud.google.com/billing/redeem
  by July 31, 2018. The coupon is valid for 90 days after redemption.
\item To estimate cloud usage costs, use Google Cloud Platform Pricing
  Calculator at
  \begin{center}
    https://cloud.google.com/products/calculator/
  \end{center}
\end{itemize}
For more details, please visit Google Cloud Platform Documentation at
https://cloud.google.com/docs/

\section*{Importing Data to BigQuery}
To access BigQuery Web UI, click on the ``BigQuery'' link on the left panel, or
navigate to https://bigquery.cloud.google.com
\begin{itemize}
\item To create a new dataset, click the down arrow icon
  \includegraphics{../arrow.png} next to your project name,
  then click ``Create new dataset''.
\item To load data to a new table, hover on the dataset name, and click on the
  plus icon \includegraphics{../plus.png}.
  \begin{itemize}
  \item Select ``Create from source'' option.
  \item Select ``File upload'' for location, and click ``Choose file'' button to
    select the file(s) to import. Make sure the ``File format'' field matches
    your file(s) to import.
  \item Choose a table name.
  \item Select ``Automatically detect'' for schema.
  \item Leave the default value for all other options.
  \end{itemize}
  The newly created table can be queried as
  \begin{center}
    $<$project\_id$>$:$<$dataset\_name$>$.$<$table\_name$>$
  \end{center}
\end{itemize}
\section*{Exporting Data from Google Cloud (Docs)}
\begin{itemize}
\item Colab notebook: \texttt{File} $\Rightarrow$ \texttt{Download .ipynb}
\item RStudio code and data: select file(s) to download in the file browser
  at bottom-right corner, then click \texttt{More} $\Rightarrow$
  \texttt{Export...}.

  You may consider compressing all files before export.
\end{itemize}

\newpage
